\documentclass[a4paper, 12pt]{article}
\usepackage[utf8]{inputenc}
\usepackage[russian]{babel} % Для русского языка
\usepackage{graphicx} % Для вставки изображений
\usepackage{float} % Для точного позиционирования графиков
\usepackage{hyperref} % Для кликабельных ссылок
\usepackage{geometry} % Настройка полей
\geometry{left=2.5cm, right=1.5cm, top=2cm, bottom=2cm}

\title{Отчет по проекту: Задачи по случайным графам}
\author{Бахурин Виктор и Стахова Екатерина}
\date{\today}

\begin{document}

\maketitle
\tableofcontents

\section{Введение}
Часть I. Исследование свойств характеристики

\section{Описание кода}
\subsection{Используемые инструменты}
\begin{itemize}
    \item Язык программирования: Python 3.10
    \item Основные библиотеки: numpy, networkx, matplotlib, scikit-learn
    \item Система контроля версий: Git (GitHub/GitLab)
    \item Дополнительные инструменты: Jupyter Notebook, PyCharm, Google Colab
\end{itemize}

\subsection{UML-диаграмма}
Мы не реализовывали свои классы. 
%\begin{figure}[H]
%    \centering
%    \includegraphics[width=0.8\textwidth]{uml_diagram.png}
%    \caption{Диаграмма классов проекта}
%    \label{fig:uml}
%\end{figure}

\subsection{Реализованные алгоритмы}
\subsubsection{$fast\_chromatic\_number$}
\begin{itemize}
    \item \textbf{Назначение}: Вычисление хроматического числа для случайного графа построенного на данной выборке.
    \item \textbf{Входные данные}: list - выборка
    \item \textbf{Выходные данные}: int - хроматическое число
    \item \textbf{Сложность}: O(nlog(n))
\end{itemize}

\section{Описание экспериментов}
\subsection{Эксперимент 1}
\subsubsection{Цель}
Исследовать, как ведет себя числовая характеристика $T$ в зависимости
от параметров распределений $θ$ и $υ$, зафиксировав размер выборки и
параметр процедуры построения графа.\\

\subsubsection{Результаты}
Характеристика $\delta(G)$ на графе KNN не подходит для определения истинности гипотез.\\
Характеристика $X$(G) на дистанционном графе показывает разные результаты для разных выборок и может использоваться для определения истинности гипотез.\\

\subsection{Эксперимент 2}
\subsubsection{Цель}
Исследовать, как ведет себя числовая характеристика T в зависимости
от параметров процедуры построения графа и размера выборки при
фиксированных значениях $\theta = \theta_0$ и $v = v_0$.\\

\subsubsection{Результаты}
Характеристика $\delta(G)$ на графе KNN не подходит для определения истинности гипотез.\\
Характеристика $X$(G) на дистанционном графе показывает похожие результаты для выборок, но в среднем график для Student-t(v) ниже, чем график для нормального распределения; можно попробовать использовать её для определения истинности.\\

\subsection{Эксперимент 3}
\subsubsection{Цель}
Построить множество $A$ в предположении $\theta = \theta_0$ и $v = v_0$ при мак-
симальной допустимой вероятности ошибки первого рода $\alpha$ = 0.055.
Оценить мощность полученного критерия.
\subsubsection{Результаты}

Удалось построить множество А.\\
Ошибка первого рода $\alpha = 0.045.$\\
Мощность полученного критерия 0.594.\\
%\section{Заключение}
%Итоговые выводы по проекту:
%\begin{itemize}
%    \item Основные достижения
%    \item Области для улучшения
%    \item Возможные направления развития
%\end{itemize}

\end{document}
