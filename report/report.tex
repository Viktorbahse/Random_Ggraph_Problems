\documentclass[a4paper, 12pt]{article}
\usepackage[utf8]{inputenc}
\usepackage[russian]{babel} % Для русского языка
\usepackage{graphicx} % Для вставки изображений
\usepackage{float} % Для точного позиционирования графиков
\usepackage{hyperref} % Для кликабельных ссылок
\usepackage{geometry} % Настройка полей
\geometry{left=2.5cm, right=1.5cm, top=2cm, bottom=2cm}

\title{Отчет по проекту: Задачи по случайным графам}
\author{Бахурин Виктор и Стахова Екатерина}
\date{\today}

\begin{document}

\maketitle
\tableofcontents

\section{Введение}
Часть I. Исследование свойств характеристики

\section{Описание кода}
\subsection{Используемые инструменты}
\begin{itemize}
    \item Язык программирования: Python 3.10
    \item Основные библиотеки: numpy, networkx, matplotlib, scikit-learn
    \item Система контроля версий: Git (GitHub/GitLab)
    \item Дополнительные инструменты: Jupyter Notebook, PyCharm, Google Colab
\end{itemize}

\subsection{UML-диаграмма}
Мы не реализовывали свои классы. 
%\begin{figure}[H]
%    \centering
%    \includegraphics[width=0.8\textwidth]{uml_diagram.png}
%    \caption{Диаграмма классов проекта}
%    \label{fig:uml}
%\end{figure}

\subsection{Реализованные алгоритмы}
\subsubsection{$fast\_chromatic\_number()$}
\begin{itemize}
    \item \textbf{Назначение}: Вычисление хроматического числа для случайного графа построенного на данной выборке.
    \item \textbf{Входные данные}: list - выборка
    \item \textbf{Выходные данные}: int - хроматическое число
    \item \textbf{Сложность}: O(nlog(n))
\end{itemize}

\subsubsection{$greedy()$}
\begin{itemize}
    \item \textbf{Назначение}: Жадное построение множества А, максимизирующие мощность критерия, при заданной допустимой ошибки первого рода.
    \item \textbf{Входные данные}: $T\_H_0$, $T\_H_1$, $\alpha$ - два набора наблюдений и максимальная допустимая ошибка первого рода.
    \item \textbf{Выходные данные}: A, $current\_error$, power - множество A, ошибка первого рода, мощность критерия. 
    \item \textbf{Сложность}: O(nlog(n))
\end{itemize}

\section{Описание экспериментов}
\subsection{Эксперимент 1}
\subsubsection{Цель}
Исследовать, как ведет себя числовая характеристика $T$ в зависимости
от параметров распределений $θ$ и $υ$, зафиксировав размер выборки и
параметр процедуры построения графа KNN.\\

\subsubsection{Результаты}

Мы получили интересный результат. График для нормального распределения выглядит хаотичнее, чем график для Student-t(ν); в графике Student-t(ν) прослеживается рост $\quad \mathbb{E}[in\_\delta(G)]$ с ростом параметра ν. И еще одно интересное наблюдение: для интересующих нас параметров распределений $v_0$ и $σ_0$ график распределения Student-t(ν) ниже графика нормального распределения.\\

\begin{figure}[H]
    \centering
    \includegraphics[width=0.9\textwidth]{EX-1-1.png}
    \caption{$\quad \mathbb{E}[in\_\delta(G)]$ для KNN графа построенного на $Normal(0,\sigma)$}
    \label{fig:uml}
\end{figure}

\begin{figure}[H]
    \centering
    \includegraphics[width=0.9\textwidth]{EX-1-2.png}
    \caption{$\quad \mathbb{E}[in\_\delta(G)]$ для KNN графа построенного на $Student-t(ν)$}
    \label{fig:uml}
\end{figure}


\subsection{Эксперимент 2}
\subsubsection{Цель}
Исследовать, как ведет себя числовая характеристика $T$ в зависимости
от параметров распределений $θ$ и $υ$, зафиксировав размер выборки и
параметр процедуры построения графа dist.\\

\subsubsection{Результаты}

Характеристика χ(G) на дистанционном графе показывает разные результаты для разных выборок. Для нормального распределения с ростом параметра $\sigma$ хроматическое число убывает, а для распределения Student-t(ν) с ростом параметра v $\chi(G)$ наоборот растет.\\

\begin{figure}[H]
    \centering
    \includegraphics[width=0.8\textwidth]{EX-2-1.png}
    \caption{$\quad \mathbb{E}[\chi(G)]$ для dist графа построенного на $Normal(0,\sigma)$}
    \label{fig:uml}
\end{figure}

\begin{figure}[H]
    \centering
    \includegraphics[width=0.8\textwidth]{EX-2-2.png}
    \caption{$\quad \mathbb{E}[\chi(G)]$ для dist графа построенного на $Student-t(ν)$}
    \label{fig:uml}
\end{figure}


\subsection{Эксперимент 3}
\subsubsection{Цель}
Исследовать, как ведет себя числовая характеристика T в зависимости
от параметров процедуры построения графа KNN и размера выборки при
фиксированных значениях $\theta = \theta_0$ и $v = v_0$.\\
\subsubsection{Результаты}
\begin{figure}[H]
    \centering
    \includegraphics[width=0.6\textwidth]{EX-3-1.png}
    \caption{$\quad \mathbb{E}[in\_\delta(G)]$ для KNN графа}
    \label{fig:uml}
\end{figure}
График для Normal выше, чем график для Student. Это может помочь в проверке истинности $H_0$ и $H_1$. \\

\subsection{Эксперимент 4}
\subsubsection{Цель}
Исследовать, как ведет себя числовая характеристика T в зависимости
от параметров процедуры построения графа и размера выборки при
фиксированных значениях $\theta = \theta_0$ и $v = v_0$.\\
\subsubsection{Результаты}
\begin{figure}[H]
    \centering
    \includegraphics[width=0.6\textwidth]{EX-4-1.png}
    \caption{$\quad \mathbb{E}[\chi(G)]$ для dist графа}
    \label{fig:uml}
\end{figure}
К сожалению, данные графики не сильно отличаются, в среднем график для Student-t(ν) ниже, чем график $Normal(0,\sigma)$.

\subsection{Промежуточный вывод}
Если обобщить результаты, полученные в предыдущих пунктах, то можно заметить, что каждая из характеристик показывает разные значения на случайных графах, построенных на распределениях $Student-t(ν)$ и нормальном распределении $Normal(0,\sigma)$. Это означает, что существует возможность использовать их для проверки истинности гипотез $H_0$ и $H_1$.

\subsection{Эксперимент 5}
\subsubsection{Цель}
Построить множество $A$ в предположении $\theta = \theta_0$ и $v = v_0$ при мак-
симальной допустимой вероятности ошибки первого рода $\alpha$ = 0.055.
Оценить мощность полученного критерия.
\subsubsection{Результаты}
Для каждой характеристики удалось построить множество A.\\
Используя характеристику $in\_\delta(G)$ на графе KNN получен следующий результат:\\
Ошибка первого рода $\alpha = 0.035.$\\
Мощность полученного критерия 0.717.\\
Используя характеристику $\chi(G)$ на графе dist получен следующий результат:\\
Ошибка первого рода $\alpha = 0.045.$\\
Мощность полученного критерия 0.594.\\
В первом случае результат значительно лучше.\\
%\section{Заключение}
%Итоговые выводы по проекту:
%\begin{itemize}
%    \item Основные достижения
%    \item Области для улучшения
%    \item Возможные направления развития
%\end{itemize}

\end{document}
